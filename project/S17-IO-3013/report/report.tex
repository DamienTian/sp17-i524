\documentclass[9pt,twocolumn,twoside]{../../styles/osajnl}
\usepackage{fancyvrb}
\journal{i524} 

\title{Music Predictive Analysis Project based on Lyrics}

\author[1,*]{Leonard Mwangi}

\affil[1]{School of Informatics and Computing, Bloomington, IN 47408, U.S.A.}

\affil[*]{Corresponding authors: lmwangi@iu.edu}

\dates{Project: S17-IO-3010, \today}

\ociscodes{Cloud, I524}

% replace this with your url in github/gitlab
\doi{Report: \url{https://github.com/lmundia/sp17-i524/tree/master/project/S17-IO-3013/report/report.pdf}\\
Code: \url{https://github.com/lmundia/sp17-i524/tree/master/project/S17-IO-3013/code}}


\begin{abstract}
Being certain that lyrics of your song will lead to the next greatest
hit would boost confidence to a lot of amateur artists who are faced
with fears of never making it thus never attempting to make good their
creativity. With Machine Learning (ML) this can be a thing of the
past, these artists would have the ability to let ML models determine
the viability of their lyrics becoming the next hit based on history
of other songs that have made it to top. Through training, the model
can certainly determine the outcome of different songs which will be
depicted in this project.\newline
\end{abstract}

\setboolean{displaycopyright}{true}

\begin{document}

\maketitle

\section{Introduction}
When faced with the decision to forward their song to a recording
company, amateur artists find it daunting due to uncertainty of
whether their song would be recorded and if it is if it will make them
wealthy.  Having ability to run the lyrics through a predictive
analysis process that would determine the viability of the song making
it would be a huge win and confidence booster to many artists. That
prediction is achievable by use of machine learning and creating a
model that takes already greatest his and trains it to determine what
makes the song successful. This would be done by analyzing the lyrics,
the locality and time of release.

In this project, we will utilize machine learning to help determine
the viability of a song becoming the next greatest hit based on the
lyrics, time of production, locality and the artist. The project will
utilize the greatest hits of all time [1] to train a model which will
then be used to analyze larger dataset of random songs [2] and provide
an in-depth analysis of the next possible hit. The project will
utilize a Hadoop cluster deployed on Chameleon Cloud using CloudMesh
to accomplish this analysis.

\section{Environment}

To achieve the analysis, Cloudmesh was used in order to allow the end
users ability to choose a cloud of their liking. Chameleon Cloud and
Jetstream were used for initial testing but changing the default cloud
on CloudMesh would allow deployment to other clouds.

\subsection{Cloudmesh}

Cloudmesh is an open source component aimed at delivering
software-defined system. It’s a collection of virtualized bare-metal
infrastructure, networks, application, systems and platform software
unifying different clouds as a Cloud Testbeds as a Service (CTaas)
\cite{cloudmesh}. Cloudmesh was developed by Professor Gregor von
Laszewski in collaboration with his team at Indiana University \cite{multipleclouds}. In this environment, cloudmesh is used to
automate installation of a four-node cluster.

\subsection{Ansible}

Ansible is also an open source platform used in automation of
deployments and tasks executions  \cite{ansible} . It has the ability to
execute scripts to a remote system thus making it easy to administer
and manage hosts farm. Ansible is written in Python thus it requires
python to be installed in the remote host for successful executions.

\subsubsection{Ansible Playboks}

Deemed as the real strength of Ansible, a playbook is a collection of
instructions that defines what Ansible will do when it connects to
each host \cite{docansible} . Playbooks are written in YAML an easy to
read data serialization language  \cite{yaml} . In this project, Ansible
is used to deploy the required components to successfully accomplished
the deployment, configuration and analysis of the defined files. 

\subsection{Apache Spark}

Apache Spark is an open-source data processing and analytics engine
designed to handle large scale dataset. Spark was originally developed
by UC Berkeley but has been adopted in the Apache software stack since
2013  \cite{amplab}. Spark has the ability to run as part of Hadoop,
Mesos, Standalone or in a cloud environment  \cite{spark}. In this
project, Spark is deployed as a standalone environment and primarily
used to analyze music data to in order to identify patterns and make
predictions.

\subsection{MongoDB}

MongoDB is a NoSQL, open source database program aimed at hosting
documents in JSON-like format. MongoDB is favorable in storing
unstructured data since it does not follow the traditional RDBMS
structure. Its performance is also paramount when coupling related
unstructured data  \cite{mongod}. For this project, MongoDB become the
ideal database application because the source data structure is unknow
and it can span in many directions where one dataset doesn’t resemble
the next. Putting such data in a structured database program would be
cumbersome.

\subsection{Python}

Python is used as the programming of choice in this project due to its
extensibility and also as a requirement for Ansible to execute
successfully.


\section{Deployment}

Environment deployment is automated as mentioned earlier using
Cloudmesh and Ansible scripts. The environment is deployed on a 4-node
cluster with the following resources.

\subsection{Resources Table}

Table \ref{tab:shape-functions} resources.

\begin{table}[htbp]
\centering
\caption{\bf resources dedicated to the environment}
\begin{tabular}{ccc}
\hline
Node & Role & Flavor  \\
\hline
$Node 1$ & $Spark$ & $m1.small$ \\
$Node 2$ & $Mesos$ & $m1.small$ \\
$Node 3$ & $MongoDB$ & $m1.medium$ \\
$Node 4$ & $Analytics$ & $m1.small$ \\
\hline
\end{tabular}
  \label{tab:shape-functions}
\end{table}

\subsection{Cloudmesh}
Code snippet below generates 4 node cluster that will be used in the
inventory file for the Ansible roles. In order to use Cloudmesh
Cluster, the following package levels or higher must be installed
\begin{itemize}
\item Cloudmesh client 4.7.2
\item Pip 8.1.2
\item Python 2.7.12  
\end{itemize}

\begin{algorithm}
\caption{cloudmesh code}\label{alg:python}
\begin{quote}
\begin{Verbatim}[numbers=left]
\$ cm cluster define --count 4 --image CC-Ubuntu16.04
\$ cm cluster use cluster-001
\$ cm cluster allocate
\$ cm cluster nodes > inventory.txt 
\end{Verbatim}
\end{quote}
\end{algorithm}

\subsubsection {format inventory file}
Define roles in the inventory file and also remove the resolved
name. After formatting the inventory file, the server roles and their
corresponding public IP address should look similar to the output
below with the right IP address.

\begin{algorithm}
\caption{Inventory file}\label{alg:python}
\begin{quote}
\begin{Verbatim}[numbers=left]
[spark\_server]
<ip address> 

[mesos\_server]
<ip address>

[mongo\_server]
<ip address>

[data\_node]
<ip address>
\end{Verbatim}
\end{quote}
\end{algorithm}

\subsection{Ansible}
After configuring deployment of virtual machines and configuring the
inventory.txt, Ansible playbooks are used to automate deployment of
the required components to the nodes defined above.
The following Ansible files are included:
ansible.cfg – used to define default configurations for Ansible including role\_path and remote user.

\begin{algorithm}
\caption{ansible config file}\label{alg:python}
\begin{quote}
\begin{Verbatim}[numbers=left]
[defaults]
roles\_path = roles
host\_key_checking=false
callback\_whitelist = profile_tasks
remote\_user=cc
\end{Verbatim}
\end{quote}
\end{algorithm}

\subsubsection{Playbooks}

The following playbooks are utilized for deployment

\begin{enumerate}
  \item predict.yml – main file for the playbooks, used to call all
    the other playbooks and define their locations. Role-based
    playbooks are contained in scripts directory of the environment.
  \item spark.yml – used to define the host, roles location and
    execution of Spark installation playbook. The playbooks redirect
    the folder structure to /roles/ansible-role-spark.
    \begin{itemize}
      \item Tasks directory - contains main.yml playbook that’s used
        to define all the tasks for Spark installation.
      \item Defaults directory – contains main.yml that defines the
        default requirements for Spark installation including the
        spark version and download site for the distribution. 
    \end{itemize}

    \item mesos.yml – used to define the host, roles location and
      execution of MongoDB installation playbook. The playbooks
      redirect the folder structure to /roles/ansible-role-mesos.
       \begin{itemize}
      \item tasks directory – contains main.yml playbook that checks
        the operating system to determine the flavor of mesos that
        would be deployed. The environment contains playbooks for
        Debian and Redhat flavors.
      \item defaults directory – contains main.yml which defines the
        version of mesos that will be downloaded and installed. 
       \end{itemize}
    \item mongo.yml – used to define the host, roles location and
      execution of MongoDB installation playbook. The playbooks
      redirect the folder structure to /roles/ansible-role-mongo.
      \begin{itemize}
      \item tasks directory - contains main.yml playbook that’s used
        to define all the tasks for MongoDB installation. The file
        also includes task to create a database and initial user for
        the Mongo environment.
      \item defaults directory – contains main.yml that defines the
        default requirements for MongoDB installation including the
        MongoD version and download site for the distribution.
      \item vars directory – contains main.yml that defines the
        variables for the environment like data path, admin account
        and password.
      \item template directory – contains mongod\_config.j2 file which
        is used to define MongoDB configurations.
      \end{itemize}

      \item data.yml - used to define the host, roles location and
        execution of dataset playbook. The playbooks redirect the
        folder structure to /roles/ansible-role-dataset.
        \begin{itemize}
        \item tasks directory - contains main.yml playbook that’s
          downloads the MillionSongs dataset and copies into to
          MongoDB. The file also contains task to create collection
          and load data into MongoDB by executing a remote file
          myfiles.py
        \end{itemize}


\end{enumerate}


\section{Conclusion}

Ability for amateur artists, artists and record labels to quick
determine the viability of a hit is paramount to their success and
missed chances due to inexperience, fear of unknows, bad song or
acting when time is not ripe can be costly. Machine learning has the
ability to change these outcomes, a well-trained model can help
determine with high accuracy where the song will end up. 


% Bibliography

\bibliography{references}
 
\end{document}
